\begin{table}[H]
\centering
\rowcolors{2}{White}{Gray!25}
\begin{tabularx}{\textwidth}{p{.20\linewidth} X X}

\hline

Tidspunkt
&
Projekt
&
Dokumentation
\\

13-21/11\newline{}Opvarmning \newline{} Præsprint
&
&
Lav \textit{case}, \textit{problemformulering}, \textit{Kravsspecifikation}, \textit{tidsplan}, \textit{mockup} og send dem til godkendelse
\\

13-21/11\newline{}Startsprint \newline{} Sprint 1
&
Lav API og Test API'en\newline{}
Lav en rudimentær frontend
&
Begynd begge rapporterne og skitser en overordnet struktur.\newline{}
Skriv første udkast til metode- og teknologiafsnittet
\\

22-28/11\newline{}Fremstød \newline{}Sprint 2
&
Gør frontend færdig\newline{}
API på nettet og sørg for kan tilgås af programmet\newline{}
Sørg for frontend kan gøres fra både tablet og web
&
Beskriv løbende program features, til produktrapporten, og erfaringer og udfordringer til procesrapporten
\\

29/11-5/12\newline{}Slutspurt \newline{} Sprint 3
&
Program testning\newline{}
Ryd op i koden og sikre der er kommenteret tilstrækkeligt\newline{}
Lav en liste over mulige \textit{features}\newline{}
Frasorter fra feature-listen, hvad der ikke kan nås, implementer, hvad der kan
&
Beskriv testene, features (både kasserede og implementerede)
\\

6-8/9\newline{}Dokumentationsmaraton \newline{} Postsprint
&
Sidste test af programmeret kører\newline{}
Eksporter filer og sikre dig, at de virker på en anden enhed\newline{}
&
Færdiggør og aflever rapporterne
\\

\hline
\end{tabularx}
\end{table}

\subsection{Principper bag tidsplanen}

Jeg har delt tidsplanen op i både projekt og dokumentering. Det ordvalg kan være lidt misvisende, fordi selvfølgelig er dokumentationen og rapporten og en del af projektet, men det er min erfaring, at jeg dokumentationen bliver gemt til sidst, hvis jeg ikke løbende sørger for at mindet mig om det. Idéen er, at få lavet så meget af dokumentationen sideløbende med projektarbejdet som muligt.\par{}
Jeg har lavet 3 egentlige sprints, også 2 halve uge, som fokusere på at forberedelsen og få lavet dokumentationen færdigt.
Hver onsdag laver en sprintplanlægningssession, hvor jeg lægger opgaver ind i et \textit{kanban-board} og hver tirsdag laver jeg et review af det sidste sprint.\par{}
Hver formiddag laver jeg en ajourføring, som i højere grad er en tjekliste snare end et egentligt \textit{Stand-up meeting}. Jeg har lavet en række spørgsmål, som jeg dagligt går igennem for at sikre, at jeg får fuldt op på loggen hver dag, at arbejdsprocessen kører optimalt og jeg fokuseret på det rigtige for dagen.

\begin{lstlisting}
Daglig ajourføring
Holdes hver arbejdsformiddag, helst inden 10. Skriv kun ja, nej eller stikord. Hvis noget skal udbydes, så gør det nede i loggen.

Gårsdagen log: Fuldendt? Tilstrækkelig? Renskrevet?

Arbejdsproces: Opfølgninger? Blokeringer? Forandres?

Hvad er dagens prioriteter?
\end{lstlisting}