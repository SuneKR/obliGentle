\projectName{} er et virtuel bræt, som udgør et alternativ til den helt almindelige bordflade, som normalt bruges til rollespil og brætspil. I modsætning til mange eksisterende \textit{Virtual TableTop (VTT)}s, er \projectName{}s designet og optimeret til at fungere i et hybridformat, hvor brugerne sidder sammen om et bord, men stadig gerne vil opnå flere af fordele ved at have et virtuel flade, og dog ønsker at holde fokus på spillet og situationen dem i mellem fremfor at blive suget helt ind i digitalt rum.\par{}
Essentielt er det også, at \projectName{} vil lade brugerne bestemme og give dem mulighederne for at definere, hvilke regler og funktion, som er implementeret i deres session.\par{}

\subsection{Funktionalitet}

\projectName{} er i sin simpleste form en digitalt flade, som brugerne kan flytte brikker på.
Endvidere har det også følgende funktioner:
\begin{itemize}
\item mulighed for at skifte brættets udseende og egenskaber
\item mulighed for at vælge, manipulere og ændre forskellige brikker
\item mulighed for at gemme, bevare og have flere sessioner
\item mulighed for at let at kunne tilføje ekstra funktioner til programmet\footnote{Altså programmet er skrevet modulært; skal ikke forstås som om brugere selv kan ændre programmet}
\end{itemize}

\subsection{Begrænsninger}

\projectName{} er optimeret til være et værktøj ved fysiske spil og ikke sit eget virtuelle rum. Derfor vil i udgangspunket ikke indeholde mulighed for, at en bruger kan gemme information fra andre brugere.\par{}
Det er også tanken, at \projectName{}, så vidt muligt, kun skal implementere regler i det omfang, at brugerne kan slå det og styrer dem selv.\par{}
Sidst men ikke mindst bliver \projectName{} kun en prototype og ikke alle funktioner vil være færdigtudviklet.

\subsection{Testkonditioner}

\begin{itemize}
\item API'en skal kunne tilgås
\item Skal virke crossplatform i både en \textit{browser} og på en \textit{tablet}
\item Skal kunne gemme sessioner
\item Skal kunne oprette nye session
\item Skal stadig fungere efter der implementeres en ny feature.
\end{itemize}